% Options for packages loaded elsewhere
\PassOptionsToPackage{unicode}{hyperref}
\PassOptionsToPackage{hyphens}{url}
%
\documentclass[
  8pt,
]{article}
\usepackage{amsmath,amssymb}
\usepackage{lmodern}
\usepackage{iftex}
\ifPDFTeX
  \usepackage[T1]{fontenc}
  \usepackage[utf8]{inputenc}
  \usepackage{textcomp} % provide euro and other symbols
\else % if luatex or xetex
  \usepackage{unicode-math}
  \defaultfontfeatures{Scale=MatchLowercase}
  \defaultfontfeatures[\rmfamily]{Ligatures=TeX,Scale=1}
\fi
% Use upquote if available, for straight quotes in verbatim environments
\IfFileExists{upquote.sty}{\usepackage{upquote}}{}
\IfFileExists{microtype.sty}{% use microtype if available
  \usepackage[]{microtype}
  \UseMicrotypeSet[protrusion]{basicmath} % disable protrusion for tt fonts
}{}
\makeatletter
\@ifundefined{KOMAClassName}{% if non-KOMA class
  \IfFileExists{parskip.sty}{%
    \usepackage{parskip}
  }{% else
    \setlength{\parindent}{0pt}
    \setlength{\parskip}{6pt plus 2pt minus 1pt}}
}{% if KOMA class
  \KOMAoptions{parskip=half}}
\makeatother
\usepackage{xcolor}
\usepackage[margin=1in]{geometry}
\usepackage{graphicx}
\makeatletter
\def\maxwidth{\ifdim\Gin@nat@width>\linewidth\linewidth\else\Gin@nat@width\fi}
\def\maxheight{\ifdim\Gin@nat@height>\textheight\textheight\else\Gin@nat@height\fi}
\makeatother
% Scale images if necessary, so that they will not overflow the page
% margins by default, and it is still possible to overwrite the defaults
% using explicit options in \includegraphics[width, height, ...]{}
\setkeys{Gin}{width=\maxwidth,height=\maxheight,keepaspectratio}
% Set default figure placement to htbp
\makeatletter
\def\fps@figure{htbp}
\makeatother
\setlength{\emergencystretch}{3em} % prevent overfull lines
\providecommand{\tightlist}{%
  \setlength{\itemsep}{0pt}\setlength{\parskip}{0pt}}
\setcounter{secnumdepth}{-\maxdimen} % remove section numbering
\usepackage{setspace}\doublespacing
\ifLuaTeX
  \usepackage{selnolig}  % disable illegal ligatures
\fi
\IfFileExists{bookmark.sty}{\usepackage{bookmark}}{\usepackage{hyperref}}
\IfFileExists{xurl.sty}{\usepackage{xurl}}{} % add URL line breaks if available
\urlstyle{same} % disable monospaced font for URLs
\hypersetup{
  pdftitle={CMSE 830 Midterm Project: PCOS Diagnosis EDA},
  pdfauthor={Angelica Louise Gacis},
  hidelinks,
  pdfcreator={LaTeX via pandoc}}

\title{CMSE 830 Midterm Project: PCOS Diagnosis EDA}
\author{Angelica Louise Gacis}
\date{2022-10-29}

\begin{document}
\maketitle

\singlespacing

\hypertarget{goal-of-the-project}{%
\subsubsection{Goal of the project}\label{goal-of-the-project}}

I chose this topic because besides being one of the main causes of
women's infertility, women with PCOS also have increased risks for type
2 diabetes and heart diseases which are some of the most common causes
of medical-related death. The exact parameters that cause PCOS are not
yet identified, but since it is a hormonal disorder, there are some
features that are suspected to help distinguish the presence or absence
of the disorder which can cause several/daily inconveniences to anyone
might have it.

\hypertarget{what-you-can-learn-from-this-app}{%
\subsubsection{What you can learn from this
app}\label{what-you-can-learn-from-this-app}}

From this app, you can learn the features of the data set that I was
able to retrieve from Kaggle. You can see the head and summary of the
data at the topmost part below the title. Next, you will find the
univariate analysis where I split it into two: categorical variables and
numerical variables. For each, I used the `PCOS diagnosis' variable for
comparing. Below that is the bivariate analysis between two numerical
variables with a hue which you can select from the categorical
variables. At the end is the correlation matrix where you can check
which variables are correlated to one another. Then, you can go back to
the univariate and bivariate plots to check the more detailed trend
between the correlated pairs. I did this in preparation of a possible
classification model for the next project.

\hypertarget{visualizations}{%
\subsubsection{Visualizations}\label{visualizations}}

I used plotly for all of my plots because they are interactive unlike
seaborn plots yet simpler unlike altair plots. For the univariate
categorical, I used a dn unstacked or dodged histogram plot. I used this
so that the user can see the imbalance in the data set among different
variables especially with our main variable which is `PCOS diagnosis'.
For the univariate numerical, I used plotly violin plots with box plots
inside to combine the power of the two plots. This is also the plot
which will show you the outliers and whether the data should be
transformed or not. I removed the background of the plots for a cleaner
view and mainly used standard primary colors with blues and reds, so
that the colors will not be distracting.

\hypertarget{preprocessing}{%
\subsubsection{Preprocessing}\label{preprocessing}}

I had two or three missing values overall from three different variables
and used the median to replace them instead of deleting them to retain
other information. I used the median so that they won't be affected by
the outliers. I also combined a two variables. For example, the number
of follicles originally had two variables, one for the right ovary, and
another for the left. However, since these two are highly correlated and
are equal most of the time, I created a new variable which is represents
the average follicle number per ovary. I also did this for follicle
size. I also removed some redundant variables such as BMI which is
redundant with the height and weight variables. Lastly, I changed the
binary variables from 0 and 1, for example to, no and yes respectively
for the plots except for the correlation matrix.

\hypertarget{additional-features}{%
\subsubsection{Additional features}\label{additional-features}}

I made versions of the data set with removed outliers, standardized
values, and both. I made a multiselectbox on the sidebar where the user
can choose how to transform the data based on what they think is
appropriate. Users can also filter the data using different variables
and values in the sidebar. Lastly, I added a slider for the correlation
matrix wherein the user can choose the range of coefficient magnitude
they want to see regardless of positive or negative signs. This will
result in a resized heatmap with only the variables producing a
correlation coefficient in the specified range remaining. The purpose of
this is because I have around 35 variables and bigger heatmaps are
harder to read, when in reality we mostly just want to extract the
variables with stronger relationships especially if this project will
lead to a classification model. All these additional features in the web
app were made to give more freedom and tools for the user to explore the
data considering that different people may discover different things
when exploring.

\end{document}
